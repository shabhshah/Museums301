\documentclass[11pt]{article}

%defines page size and margins
\usepackage{geometry}
\geometry{
    letterpaper,
    left=1in,
    right=1in,
    top=1in,
    bottom=1in,
}

%Sets spacing for entire document
\usepackage{setspace}
\doublespacing

%Package for reducing space in between list items
\usepackage{enumitem}

%Math symbols
\usepackage{gensymb}

%Links n shit
\usepackage{hyperref}

%dope tables
\usepackage{array}
\usepackage{float}

%Image path
\usepackage{graphicx}
\graphicspath{ {../Images/} }

%Makes lists easier
\newcommand{\ol}[1]{\begin{enumerate}[noitemsep, topsep=0pt]#1\end{enumerate}}
\newcommand{\ul}[1]{\begin{itemize}[noitemsep, topsep=0pt]#1\end{itemize}}
\newcommand{\li}[1]{\item#1}

%Bibliography
\newenvironment{bib}[1]
  {\begin{list}
          {}
          {\setlength{\itemindent}{-#1}
           \setlength{\leftmargin}{#1}
           \setlength{\itemsep}{0pt}
           \setlength{\parsep}{\parskip}
           \setlength{\topsep}{\parskip}
           }
    \setlength{\parindent}{-#1}
    \item[]
  }
  {\end{list}}

\begin{document}
\singlespacing
{\Large\noindent Rishabh Shah

\noindent 4655 4192

\noindent October 25, 2017

\noindent Museums 301 Midterm Question Three}

\subsection*{The film \textit{Robbing the Cradle of Civilization}, about the 2003 looting of the Iraqi National Museum, provides a case study for exploring ethical issues surrounding ownership and care of the world’s cultural heritage. According to this film, who are the ``major players'' in the world’s antiquities market and what motivates each of these groups to participate? Specifically, what role do museums play in this network? Finally, in your opinion, which of these groups is ultimately responsible for ``robbing the cradle of civilization''? Use specific examples from the film, class readings, or lectures to support your answer.}

\doublespacing
When the United States led coalition invaded Iraq in 2003, it was the beginning of a long and bloody conflict that still continues today. The invasion was largely based on the claim that Iraq was in possession of weapons of mass destruction, along with the goal of freeing the Iraqi people. The invasion began in mid-March of 2003 and coalition forces quickly reached and captured the capital Baghdad by early-April.\footnote{``Iraq War 2003-2011,'' \textit{Britannica}, accessed October 24, 2017, \url{https://www.britannica.com/event/Iraq-War}.} During the Battle of Baghdad, the Iraqi National Museum was looted consecutively for three days, with thousands of pieces of history systemically stolen. The history of Iraq is extremely old, as it is the site of one of the first civilizations, ancient Mesopotamia. Many of the stolen pieces were thousands of years old and are worth thousands of dollars. 

According to the film \textit{Robbing the Cradle of Civilization}, the ``major players'' in the world's antiquities market are the looters, private collectors, governments, armies, and museums.\footnote{\textit{Robbing the Cradle of Civilization}, directed by Robin Benger (2003; Ottawa, Canada: Canadian Broadcasting Corporation, 2003.), TV.} Looters are the first to interact with items to be brought into the world's antiquities market. Lack of economic growth causes many to turn to less than ideal ways of surviving. In areas with high concentrations of antiquities, slow economic growth or even economic decline can lead to looting of these items. These items are sold to private collectors, who pay large amounts of money to add antiquities to their exhibits. Private collectors also play a major role in the market, as without them, there would be limited incentive for looters to pillage their own history. Additionally, governments also play a part in the global antiquities trade. For example, in Israel, many stolen Iraqi seals are openly sold in government approved shops.\footnote{\textit{Ibid.}} These activities are sanctioned by the government of Israel, and is an outlet for stolen goods from Iraq and other Middle Eastern countries to the private collectors. In the case of the Iraqi National Museum, the coalition forces were also a player in how the items entered the antiquities market. When the coalition invaded Iraq, it was partly under the guise that they were there to free the Iraqi people. However, upon entering Baghdad, the coalition did not protect the Iraqi National Museum until a week after the looting began. The research director of the museum, Donny George asked American forces to help protect the museum immediately after the looting began, but unfortunately they were unable to help as they did not have the necessary orders.\footnote{\textit{Ibid.}} By the time the orders were given, it was too late as the looting was already over. By creating the conditions for the looting to occur (invading Baghdad), and not protecting the Iraqi National Museum, the coalition forces were also implicit in the addition of Iraqi antiquities to the market. Finally, museums also play a role in the antiquities market due to their ability to ``wash'' the looted status off of items.\footnote{\textit{Ibid.}} While most museums cannot own stolen art, they are able to exhibit private collections, which can contain stolen art. Once stolen art is displayed in a museum, it is perceived as clean by the public.

Museums play a crucial role in the network of stolen antiquities by displaying private collections which contain looted items. A former employee of the Royal Ontario Museum, Elie Borowski, had a massive private collection of Mesopotamian items. This collection was placed on display in 1985 in the Royal Ontario Museum as an exhibit.\footnote{\textit{Ibid.}} Benett Bronson, from Chicago, noticed that one of the seals in the collection had been in the Iraqi National Museum.\footnote{\textit{Ibid.}} However, when Elie was confronted about the looted seal, he denied knowing about it.\footnote{\textit{Ibid.}} Even though the museum technically does not own the looted antiquities, it still is displaying them for the museum's audience to see. This attitude that museums hold towards the history of items in private collections is dangerous, as it leads to normalized displays of looted antiquities.

Although it is difficult to identify one specific party responsible, ultimately, the looters are most responsible for ``robbing the cradle of civilization.'' They are the ones taking the antiquities from historical sites and putting them into the market. Without the looters, the private collectors would have no supply to buy antiquities from and in turn, private collections would be smaller and less likely to be displayed in museums. Unfortunately, it is also the looters who are least likely to stop what they are doing. The majority of looters are not looting for the enjoyment, rather they loot to make money to survive. This paradox highlights the complexity and nuances of the global antiquities market, as the ones most responsible for ``robbing the cradle of civilization'' are doing it simply because they have to.

In order for change to occur in the global antiquities market, many fundamental things need to be altered. Firstly, in order to stop looting, other economic opportunities need to be made available. This will remove the root cause of most looting. Secondly, international laws regarding the antiquities need to be strengthened, resulting in the items being properly tracked and returned to the rightful owners. Thirdly, museums need to pay closer attention to the history of the items they are displaying, regardless if they own them or not.

\newpage
\singlespacing
\begin{center}
{\large\textit{Works Cited}}
\end{center}

\begin{bib}{2em}
``Iraq War 2003-2011.'' \textit{Britannica}. Accessed October 24, 2017. \url{https://www.britannica.com/event/Iraq-War}. \\

\textit{Robbing the Cradle of Civilization}. Directed by Robin Benger. 2003. Ottawa, Canada: Canadian Broadcasting Corporation, 2003. TV.
\end{bib}
\end{document}