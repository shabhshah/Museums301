\documentclass[11pt]{article}

%defines page size and margins
\usepackage{geometry}
\geometry{
    letterpaper,
    left=1in,
    right=1in,
    top=1in,
    bottom=1in,
}

%Sets spacing for entire document
\usepackage{setspace}
\singlespacing

%Package for reducing space in between list items
\usepackage{enumitem}

%tables
\usepackage{float}
\usepackage{longtable}
\usepackage{array}

%Math symbols
\usepackage{gensymb}

%Makes lists easy
\newcommand{\ol}[1]{\begin{enumerate}[noitemsep]#1\end{enumerate}}
\newcommand{\ul}[1]{\begin{itemize}[noitemsep]#1\end{itemize}}
\newcommand{\li}[1]{\item#1}

\begin{document}
{\large \noindent Rishabh Shah

\noindent November 3, 2017

\noindent 4655 4192

\noindent Digital Curation Project Part Two}

\begin{center}
	\begin{longtable}{>{\raggedleft\arraybackslash}p{1in} >{\raggedright\arraybackslash}p{5.5in}}
		Boots & Necessary for backcountry adventures, boots have been around for centuries. These boots are the TPS 520 by Asolo. Made of leather, they feature a waterproof and breathable Gore-Tex lining. These boots can be used in all seasons. Purchased in April of 2017, these boots have already seen heavy use in the White Mountains of New Hampshire and the Garhwal Himalayas in Uttarakhand, India. These boots have broken in nicely providing comfort to the wearer.\\
		Pants & Used in all seasons, pants provide protection from wind, water, brush, the sun, and many other elements faced in the backcountry. These specific pants are made of a synthetic material, allowing for quick drying times. Designed and sold by Costco, these pants are highly durable and have been used in all conditions from deep winter to the heat of the summer.\\
		Socks & Socks are just as important as boots when it comes to staying energetic and warm in the backcountry. Warm socks boost morale in the worst of conditions. These specific socks are manufactured in America and consist of mostly Merino Wool, a unique type of wool found on sheep from New Zealand. Merino wool features many properties such as anti-odor, durability, and warmth. The thicker the sock, the warmer it is.\\
		Hat & Hats are a staple of outdoor culture as they protect the wearer from the sun, rain, wind, and keep warmth in. This hat was purchased at the Randolph Mountain Club's (RMC) Gray Knob Cabin on Mt. Adams in the White Mountains of New Hampshire. Worn in mostly 3-season conditions, it does not do well in cold weather due to its cotton construction. The RMC maintains the trail system on the Northern Presidential Range of the White Mountains.\\
		Hard shell & When the wind starts blowing or the rain starts falling, a hard shell is the one jacket that will keep you dry and warm. This hard shell, made by The North Face, features a tough outer fabric which resists abrasions extremely well. The minimal pockets and zippers maintain simplicity and lowers the chance of failure in the field. While mostly used in the winter, hard shells can be used all year round.\\
		Water bottle & Hydration can be the difference between life and death in the backcountry. A water bottle like this allows the hiker to efficiently carry water. Nalgene brand water bottles are able to hold boiling water and can freeze solid overnight without cracking. This water bottle has been used for a long time, as shown by the multitude of stickers, and the numerous scratches and marks.\\
		Backpack & This large backpack can hold enough gear for many days in the backcountry. Featuring lots of technical features, this pack has been used in all conditions- day hikes all the way to winter expeditions. However, the large size comes with its own downsides. The large size does contribute to a heavier base weight, which can slow the hiker down.\\
		Sleeping bag & This Western Mountaineering Antelope MF features continuous down baffling and a draft tube to ensure warmth to below 5 degrees. This lightweight bag can keep the sleeper warm with only two pounds of down. This bag has been used mostly in the winter time, and has also been used in the Garhwal Himalayas.\\
		Sleeping pad & Sleeping pads insulate the sleeping bag from the ground. Without a sleeping pad, the body heat trapped in the sleeping bag will transfer to the ground. With a sleeping pad, this heat transfered is slowed, allowing for more comfortable nights camping.\\
		Tent & The most popular form of backcountry shelter, tents are staples in almost all outdoor gear set ups. This specific tent is a three-season, three person tent. However, it has been used in winter conditions without problems. The bright orange exterior helps in finding the tent when returning from nighttime bathroom runs.\\
		Stove and fuel bottle & This stove features the timeless design of liquid fuel powered camping stove. The advantage of a liquid fuel camping stove is the operating temperature range extends well below 0 degrees, unlike other stove systems which become ineffective at temperatures below 20 degrees. The fuel bottle features a layer of duct tape on the outside. This is used to mitigate frostbite in the winter time from grabbing the metal directly without gloves on.\\
		Water filter & It is important to filter all water in the backcountry before drinking, unless the source of the water can be verified. This filter utilizes a pump action to force water through a filter, which removes debris and bacteria. However, if the temperature is predicted to drop below freezing, ceramic filters are not recommended as the filter itself can crack due to the cold.\\
		Down jacket & A down jacket is a perfect piece to wear when taking a break from aerobic activity due to its immense warmth to weight ratio. This jacket has been owned for many years, providing countless memories skiing or hiking in less than ideal conditions but still enjoying it because of the warmth the jacket provided.\\
		Microspikes & Microspikes are used in winter to provide additional traction to the wearer. They are made of steel and rubber and feature 3/8" long teeth which dig into snow, ice, and rocks above and below treeline. The advantage of Microspikes is that they can attach to almost any footwear immediately, making them a great initial investment.\\
		Ice axe & The ice axe is the hallmark of general mountaineering. Used mostly for navigating steep sections of snow and ice, and self arrests, ice axes are crucial in areas where conditions are less than optimal. Glissading, a technique that utilizes an ice axe, is usually the fastest down the mountain in the winter. However, if not used correctly, the ice axe can injure the hiker glissading.\\
	\end{longtable}
\end{center}
\end{document}
