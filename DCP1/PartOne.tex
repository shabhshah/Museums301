\documentclass[10pt]{article}

%defines page size and margins
\usepackage{geometry}
\geometry{
    letterpaper,
    left=1in,
    right=1in,
    top=1in,
    bottom=1in,
}

%Sets spacing for entire document
\usepackage{setspace}
\singlespacing

%Package for reducing space in between list items
\usepackage{enumitem}

%Math symbols
\usepackage{gensymb}

%Image path
\usepackage{graphicx}
\graphicspath{ {/} }

%Makes lists easy
\newcommand{\ol}[1]{\begin{enumerate}[noitemsep]#1\end{enumerate}}
\newcommand{\ul}[1]{\begin{itemize}[noitemsep]#1\end{itemize}}
\newcommand{\li}[1]{\item#1}

\begin{document}
{\large \noindent Rishabh Shah

\noindent September 29, 2017

\noindent 4655 4192

\noindent Digital Curation Project Part One}

\section*{Narrative}

My narrative for the Digital Curation Project is to tell my story through the White Mountains of New Hampshire. I have been hiking in the White Mountains since 2012, when I climbed my first 4000' summit on Middle Tripyramid. However, I did not seriously start hiking in the White Mountains until the spring of 2016. Since then, I have logged over 200 miles hiking in the White Mountains, and they have had a significant impact on me. I plan on having the museum structured in a way which takes the reader through the gear required to do the type of overnight hikes I have done and will continue to do. The objects listed below are in no particular order, however, I will arrange the museum in a way that the visitor will be guided from the gear required for the most basic hikes, to the gear required for more advanced hikes, including winter backpacking.

\section*{Objects}
\ol{%
	\li{Backpack}
	\ul{%
		\li{Used to transport gear for the hike, food, and water. The one I will be selecting will the larger backpack used on overnight hikes.}
		}
	\li{Sleeping bag}
	\ul{%
		\li{Used to sleep in on overnight backpacking trips. Uses down as insulation}
		}
	\li{Tent}
	\ul{%
		\li{Used to sleep in on overnight backpacking trips. Can fit three people.}
		}
	\li{Microspikes}
	\ul{%
		\li{Attached onto the bottom of boots to add traction in snowy and icy trail conditions. Utilizes 3/8'' long steel spikes to dig into the ground.}
		}
	\li{Ice axe}
	\ul{%
		\li{Used to self arrest on icy or snowy slopes. Can also be used to get up said slopes.}
		}
	
	\li{Hat}
	\ul{%
		\li{Used to keep sweat off of the forehead, block the sun from the face, and also traps heat to maintain warmth.}
		}
	\li{Pants}
	\ul{%
		\li{Made of synthetic material to ensure quick drying speeds and moisture wicking capabilities.}
		}
	\li{Socks}
	\ul{%
		\li{Made of Merino wool to wick moisture away from feet, reduces odors, and are very warm.}
		}
	\li{Down jacket}
	\ul{%
		\li{Used when not engaging in highly aerobic activities to maintain body heat. Does not stop water.}
		}
	\li{Hard shell}
	\ul{%
		\li{Used whenever water or high winds are present. Is completely waterproof and windproof.}
		}

	\li{Water bottle}
	\ul{%
		\li{Used to store water. Made by Nalgene. Can withstand being frozen. Can carry boiling water.}
		}
	\li{Water filter}
	\ul{%
		\li{Used to purify water. Used only in above freezing temperatures. Will break in subfreezing temperatures. Pump action.}
		}
	\li{Stove and fuel bottle}
	\ul{%
		\li{Used to cook food, boil water, and melt snow. Runs off of white gas, and will reliably work when wet and down to -40\degree F.}
		}
	\li{Boots}
	\ul{%
		\li{Used to protect feet from the environment. Are waterproof and made of leather.}
		}
	\li{Sleeping pad}
	\ul{%
		\li{Used to insulate body from cold ground when sleeping. Made of foam. Sleeping bags will not insulate well without a sleeping pad.}
		}
}
\end{document}