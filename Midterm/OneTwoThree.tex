\documentclass[11pt]{article}

%defines page size and margins
\usepackage{geometry}
\geometry{
    letterpaper,
    left=1in,
    right=1in,
    top=1in,
    bottom=1in,
}

%Sets spacing for entire document
\usepackage{setspace}
\doublespacing

%Package for reducing space in between list items
\usepackage{enumitem}

%Math symbols
\usepackage{gensymb}

%dope tables
\usepackage{array}
\usepackage{float}

%Links
\usepackage{hyperref}

%Image path
\usepackage{graphicx}
\graphicspath{ {../Images/} }

%Makes lists easier
\newcommand{\ol}[1]{\begin{enumerate}[noitemsep, topsep=0pt]#1\end{enumerate}}
\newcommand{\ul}[1]{\begin{itemize}[noitemsep, topsep=0pt]#1\end{itemize}}
\newcommand{\li}[1]{\item#1}

\begin{document}
\singlespacing
{\Large\noindent Rishabh Shah

\noindent 4655 4192

\noindent October 25, 2017

\noindent Museums 301 Midterm}
\section*{I}
\singlespacing
\subsection*{Name at least two ways in which the deaccessioning of George Catlin paintings from Chicago’s Field Museum was controversial.}

\doublespacing
When the Field Museum in Chicago deaccessioned the paintings of George Catlin, the museum received money in turn for selling them. However, the controversy did not surround the deaccessioning of the paintings. Instead, the controversy surrounded what the Field Museum did with the money. Instead of spending the money on collecting, preserving, and caring for objects, the administration decided to pay salaries of several staff members. While not illegal, using money from deaccessioned collections for uses other than the collection, preservation, and care of objects is an ethical dilemma. Many in the industry disagreed with the Field Museum's decision to spend the money in a nonstandard method. Another source of controversy surrounding the deaccessioning of Catlin's paintings deals with the expectation of donors. Many benefactors of collections expect that the museums they contribute to keep their objects forever. Potential donors may see the deaccessioning of Catlin's collection as a signal that their collections could be sold if the museum becomes strapped for cash again.

\singlespacing
\subsection*{Provide two primary reasons why museums should exercise caution when considering deaccessioning items from their collection.}

\doublespacing
When deaccessioning items from collections, museums must exercise extreme caution. One reason is that museums are charged with certain duties, among them, collecting and preserving objects. When a museum deaccessions objects or collections from its care, it must take care that the party which is receiving the items is in line with the mission statement of the museum. If the museum can not guarantee this, then they are at risk of violating their mandate to collect and preserve objects, as they would be responsible if anything harmed the objects. Another reason is due to the generally accepted standard that museums spend the money they receive from deaccessioned objects to collect and preserve objects. If the museum does not spend the money in a correct manner, it can lead to negative opinions of the museum by the public. This hurts the museum, and can result in reduced attendance, donations, and public image.

\singlespacing
\subsection*{Define the term ``repatriation'' with regard to museum collections and cite at least two instances/examples in which repatriation was discussed in class.}

\doublespacing
Repatriation is when a stolen or looted object is returned to a large institution or country. One example deals with the Elgin Marbles. These marbles were taken from the Parthenon in Greece in the early 1800s, and put on display in the British Museum shortly after. While the marbles are still on display in the British Museum, the government and people of Greece have requested the marbles be returned, or repatriated, to the rightful owners, the people of Greece. Another example is with the Native American Graves Protection and Repatriation Act. Many museums had remains of Native Americans on display, to which many tribes claimed to be their property and an integral part of their heritage. The NAGPRA was passed in an effort to return the remains to the tribes.

\singlespacing
\subsection*{What is the essential difference between a collection of objects owned by an individual collector and a museum?}

\doublespacing
A collection of objects owned by an individual is different than a museum is due to the purpose of each. A museum is charged with many duties: educating the community and serving the public; collecting, preserving, and cataloging objects; and saving cultural and natural heritages, among many others. However, a private collection does not need to meet any of these duties, as it is simply a collection of objects a private individual owns. A collection in a museum will be used according to the mission statement of the museum, which contains nuanced versions of the duties all museums are charged with. If executed correctly, the museum, and its collections, will have a positive impact upon its community, while private collections do not have an obligation to do so.

\singlespacing
\subsection*{Reformer John Cotton Dana is credited with many innovations in American museums. What was it specifically that Dana was reacting against.}

\doublespacing
John Cotton Dana was a professional librarian who created the Public Museum of Newark, NJ. With the creation of this museum, Dana also developed a new model of what museums could be. According to Dana, collections had to directly aid members of the local community. He focused on tying the exhibitions at the Public Museum of Newark closely to the needs and interests of the local community. Dana's museum and the exhibits inside it, which were designed as user oriented, were shared freely with the community.

Dana did not see value in private collections, or collections which did not put the local community first. He strongly opposed the Metropolitan Museum of Art located in New York City. This was due to the Met not putting the community first. Additionally, Dana held a strong dislike towards the large staircase leading to the entrance of the Met. According to him, the grandeur of the staircase did not welcome all members of the community.

\section*{II}
\singlespacing
\subsection*{Using the article by Fitz Gibbon as your starting point, discuss the debate over ownership of the Elgin/Parthenon Marbles, addressing the following points: What is the origin and specific nature of the conflict? What arguments for ownership have been advanced by countries involved? Why might ownership of the Marbles matter to the parties involved?}

\doublespacing
One of the most visited exhibitions at The British Museum are the Eglin Marbles. These Classical Greek marble sculptures have been on display since 1817.\footnote{``Parthenon Sculptures,'' \textit{The British Museum,} accessed October 24, 2017, \url{http://www.britishmuseum.org/about_us/news_and_press/statements/parthenon_sculptures.aspx}.} Named after Thomas Bruce, the Earl of Elgin and Ambassador to the Ottoman Empire, the marbles were taken from the Parthenon in Athens and shipped to London in the early 1800s.\footnote{\textit{Ibid.}} After financial problems caused Elgin to sell the marbles to Parliament in 1816, they were put on display in the British Museum.\footnote{Kate Fitz Gibbon, \textit{Who Owns the Past?: Cultural Policy, Cultural Property, and the Law} (New Brunswick, NJ: Rutgers University Press, 2005), 113.} Since Elgin began acquiring the marbles, they have been a source of controversy over who the rightful owner is, and how to move forward. Greece believes that the marbles should be returned, but the English consider the marbles to be their own.

Elgin arrived in Athens in 1800 with the ``intent to take... measurements and make drawings and plaster casts from the Parthenon.''\footnote{Gibbon, 110.} However, once Elgin moved to Constantinople, his men started to ``dismantle a substantial portion of the finest... remaining decorative elements of the Parthenon.''\footnote{Gibbon, 111.} Before heading back to England, Elgin was arrested by the French due to his role in driving them from Egypt and because he had secured the marbles before Napoleon was able to. During this time, Elgin's mother collected the marbles in a ``shedlike building.''\footnote{\textit{Ibid.}} When he returned to England deep in debt, he attempted to sell the marbles to Parliament for The British Museum.\footnote{\textit{Ibid.}} However, the debate over the rightful ownership had already started. Some in Parliament wanted to keep the marbles for their protection and preservation, while a minority argued to return the marbles to the Ottoman government.\footnote{Gibbon, 112.} After fierce debates, it was decided that Parliament would buy the marbles from Elgin, and The British Museum placed the marbles on display.

In 1982, the Greek Minister of Culture called on England to return the marbles to Greece.\footnote{Gibbon, 113.} That same year, Greece formally requested that Britain return the marbles. However, the request was declined, with the trustees of The British Museum ``[stating] several times that their duty to the public and interest in preserving the sculptures precluded any consideration of return of the marbles.''\footnote{Gibbon, 114.} Even though the Greek government's request was denied, it has continued to request that the British return the marbles.

The legal arguments for rightful ownership begin with the authorization by the Ottoman Empire for Elgin to take the marbles. At first, local Ottoman officials harassed Elgin's men, but after Elgin obtained a ``\textit{firman}'', the men were able to work without hindrance.\footnote{\textit{Ibid.}} This \textit{firman} was used to legitimize the purchase of the marbles by Parliament, but the version used was a translation.\footnote{Gibbon, 115.} The translation did not have an official seal, and the original document has not been found. Without the original document, it is difficult to tell what authorization Elgin had from the government to excavate, and how much he could take. However, the statute of limitations for the return of stolen property has run out, as Greece gained independence in 1828, but did not request the marbles back until 1983.\footnote{Gibbon, 116.}

The ethical arguments for rightful ownership are far more applicable, as the Greek government has stopped pursuing the legal route for repatriation of the marbles. According to Greece, ``the sculptures are Greek and therefore belong in Greece.''\footnote{\textit{Ibid.}} Additionally, many Greeks feel deprived of their heritage and culture since they are not easily able to see the marbles, take part in the ``preservation, study, and enjoyment'' of their own history, nor use the marbles for monetary gain.\footnote{Gibbon, 117.} 

According to the British, if the marbles were not removed, the Parthenon's history would not be preserved. The British back this claim up with ``evidence that the Parthenon... [was] damaged by the depredations of visiting tourists, local antiquities dealers, and agents of other countries.''\footnote{\textit{Ibid.}} However, between 1937 and 1938, The British Museum failed to protect the marbles by allowing an abrasive cleaning to take place.\footnote{Gibbon, 118.} Another argument the British put forward is the marbles are an integral part of The British Museum. ``Almost all of the five million visitors a year'' of the museum see the marbles.\footnote{\textit{Ibid.}} If the marbles were returned to Greece, much fewer people would see them, reducing access for large amounts of people. In spite of this, many more Greeks would have access to view the marbles.

For over two hundred years, the Elgin Marbles have been at the source of one of the largest ownership controversies. There is no clear answer as to who has a legal and moral right to marbles due to the nuances of the situation, and both the British and the Greeks think they are correct. The Greeks have a claim to the marbles from a ``cultural nationalist'' viewpoint, while the British have a claim from a ``internationalist'' viewpoint.\footnote{Gibbon, 116.}

\section*{III}
\singlespacing
\subsection*{The film \textit{Robbing the Cradle of Civilization}, about the 2003 looting of the Iraqi National Museum, provides a case study for exploring ethical issues surrounding ownership and care of the world’s cultural heritage. According to this film, who are the ``major players'' in the world’s antiquities market and what motivates each of these groups to participate? Specifically, what role do museums play in this network? Finally, in your opinion, which of these groups is ultimately responsible for ``robbing the cradle of civilization''? Use specific examples from the film, class readings, or lectures to support your answer.}

\doublespacing
When the United States led coalition invaded Iraq in 2003, it was the beginning of a long and bloody conflict that still continues today. The invasion was largely based on the claim that Iraq was in possession of weapons of mass destruction, along with the goal of freeing the Iraqi people. The invasion began in mid-March of 2003 and coalition forces quickly reached and captured the capital Baghdad by early-April.\footnote{``Iraq War 2003-2011,'' \textit{Britannica}, accessed October 24, 2017, \url{https://www.britannica.com/event/Iraq-War}.} During the Battle of Baghdad, the Iraqi National Museum was looted consecutively for three days, with thousands of pieces of history systemically stolen. The history of Iraq is extremely old, as it is the site of one of the first civilizations, ancient Mesopotamia. Many of the stolen pieces were thousands of years old and are worth thousands of dollars. 

According to the film \textit{Robbing the Cradle of Civilization}, the ``major players'' in the world's antiquities market are the looters, private collectors, governments, armies, and museums.\footnote{\textit{Robbing the Cradle of Civilization}, directed by Robin Benger (2003; Ottawa, Canada: Canadian Broadcasting Corporation, 2003.), TV.} Looters are the first to interact with items to be brought into the world's antiquities market. Lack of economic growth causes many to turn to less than ideal ways of surviving. In areas with high concentrations of antiquities, slow economic growth or even economic decline can lead to looting of these items. These items are sold to private collectors, who pay large amounts of money to add antiquities to their exhibits. Private collectors also play a major role in the market, as without them, there would be limited incentive for looters to pillage their own history. Additionally, governments also play a part in the global antiquities trade. For example, in Israel, many stolen Iraqi seals are openly sold in government approved shops.\footnote{\textit{Ibid.}} These activities are sanctioned by the government of Israel, and is an outlet for stolen goods from Iraq and other Middle Eastern countries to the private collectors. In the case of the Iraqi National Museum, the coalition forces were also a player in how the items entered the antiquities market. When the coalition invaded Iraq, it was partly under the guise that they were there to free the Iraqi people. However, upon entering Baghdad, the coalition did not protect the Iraqi National Museum until a week after the looting began. The research director of the museum, Donny George asked American forces to help protect the museum immediately after the looting began, but unfortunately they were unable to help as they did not have the necessary orders.\footnote{\textit{Ibid.}} By the time the orders were given, it was too late as the looting was already over. By creating the conditions for the looting to occur (invading Baghdad), and not protecting the Iraqi National Museum, the coalition forces were also implicit in the addition of Iraqi antiquities to the market. Finally, museums also play a role in the antiquities market due to their ability to ``wash'' the looted status off of items.\footnote{\textit{Ibid.}} While most museums cannot own stolen art, they are able to exhibit private collections, which can contain stolen art. Once stolen art is displayed in a museum, it is perceived as clean by the public.

Museums play a crucial role in the network of stolen antiquities by displaying private collections which contain looted items. A former employee of the Royal Ontario Museum, Elie Borowski, had a massive private collection of Mesopotamian items. This collection was placed on display in 1985 in the Royal Ontario Museum as an exhibit.\footnote{\textit{Ibid.}} Benett Bronson, from Chicago, noticed that one of the seals in the collection had been in the Iraqi National Museum.\footnote{\textit{Ibid.}} However, when Elie was confronted about the looted seal, he denied knowing about it.\footnote{\textit{Ibid.}} Even though the museum technically does not own the looted antiquities, it still is displaying them for the museum's audience to see. This attitude that museums hold towards the history of items in private collections is dangerous, as it leads to normalized displays of looted antiquities.

Although it is difficult to identify one specific party responsible, ultimately, the looters are most responsible for ``robbing the cradle of civilization.'' They are the ones taking the antiquities from historical sites and putting them into the market. Without the looters, the private collectors would have no supply to buy antiquities from and in turn, private collections would be smaller and less likely to be displayed in museums. Unfortunately, it is also the looters who are least likely to stop what they are doing. The majority of looters are not looting for the enjoyment, rather they loot to make money to survive. This paradox highlights the complexity and nuances of the global antiquities market, as the ones most responsible for ``robbing the cradle of civilization'' are doing it simply because they have to.

In order for change to occur in the global antiquities market, many fundamental things need to be altered. Firstly, in order to stop looting, other economic opportunities need to be made available. This will remove the root cause of most looting. Secondly, international laws regarding the antiquities need to be strengthened, resulting in the items being properly tracked and returned to the rightful owners. Thirdly, museums need to pay closer attention to the history of the items they are displaying, regardless if they own them or not.
\end{document}