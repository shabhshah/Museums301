\documentclass[11pt]{article}

%defines page size and margins
\usepackage{geometry}
\geometry{
    letterpaper,
    left=1in,
    right=1in,
    top=1in,
    bottom=1in,
}

%Sets spacing for entire document
\usepackage{setspace}
\doublespacing

%Package for reducing space in between list items
\usepackage{enumitem}

%Math symbols
\usepackage{gensymb}

%dope tables
\usepackage{array}
\usepackage{float}

%Image path
\usepackage{graphicx}
\graphicspath{ {../Images/} }

%Makes lists easier
\newcommand{\ol}[1]{\begin{enumerate}[noitemsep, topsep=0pt]#1\end{enumerate}}
\newcommand{\ul}[1]{\begin{itemize}[noitemsep, topsep=0pt]#1\end{itemize}}
\newcommand{\li}[1]{\item#1}

\begin{document}
\begin{singlespacing}
{\Large\noindent Rishabh Shah

\noindent 4655 4192

\noindent October 25, 2017

\noindent Museums 301 Midterm Question One}
\end{singlespacing}

\begin{singlespacing}
\subsection*{Name at least two ways in which the deaccessioning of George Catlin paintings from Chicago’s Field Museum was controversial.}
\end{singlespacing}

When the Field Museum in Chicago deaccessioned the paintings of George Catlin, the museum received money in turn for selling them. However, the controversy did not surround the deaccessioning of the paintings. Instead, the controversy surrounded what the Field Museum did with the money. Instead of spending the money on collecting, preserving, and caring for objects, the administration decided to pay salaries of several staff members. While not illegal, using money from deaccessioned collections for uses other than the collection, preservation, and care of objects is an ethical dilemma. Many in the industry disagreed with the Field Museum's decision to spend the money in a nonstandard method. Another source of controversy surrounding the deaccessioning of Catlin's paintings deals with the expectation of donors. Many benefactors of collections expect that the museums they contribute to keep their objects forever. Potential donors may see the deaccessioning of Catlin's collection as a signal that their collections could be sold if the museum becomes strapped for cash again.

\begin{singlespacing}
\subsection*{Provide two primary reasons why museums should exercise caution when considering deaccessioning items from their collection.}
\end{singlespacing}

When deaccessioning items from collections, museums must exercise extreme caution. One reason is that museums are charged with certain duties, among them, collecting and preserving objects. When a museum deaccessions objects or collections from its care, it must take care that the party which is receiving the items is in line with the mission statement of the museum. If the museum can not guarantee this, then they are at risk of violating their mandate to collect and preserve objects, as they would be responsible if anything harmed the objects. Another reason is due to the generally accepted standard that museums spend the money they receive from deaccessioned objects to collect and preserve objects. If the museum does not spend the money in a correct manner, it can lead to negative opinions of the museum by the public. This hurts the museum, and can result in reduced attendance, donations, and public image.

\begin{singlespacing}
\subsection*{Define the term ''repatriation'' with regard to museum collections and cite at least two instances/examples in which repatriation was discussed in class.}
\end{singlespacing}

Repatriation is when a stolen or looted object is returned to a large institution or country. One example deals with the Elgin Marbles. These marbles were taken from the Parthenon in Greece in the early 1800s, and put on display in the British Museum shortly after. While the marbles are still on display in the British Museum, the government and people of Greece have requested the marbles be returned, or repatriated, to the rightful owners, the people of Greece. Another example is with the Native American Graves Protection and Repatriation Act. Many museums had remains of Native Americans on display, to which many tribes claimed to be their property and an integral part of their heritage. The NAGPRA was passed in an effort to return the remains to the tribes.

\begin{singlespacing}
\subsection*{What is the essential difference between a collection of objects owned by an individual collector and a museum?}
\end{singlespacing}

A collection of objects owned by an individual is different than a museum is due to the purpose of each. A museum is charged with many duties: educating the community and serving the public; collecting, preserving, and cataloging objects; and saving cultural and natural heritages, among many others. However, a private collection does not need to meet any of these duties, as it is simply a collection of objects a private individual owns. A collection in a museum will be used according to the mission statement of the museum, which contains nuanced versions of the duties all museums are charged with. If executed correctly, the museum, and its collections, will have a positive impact upon its community, while private collections do not have an obligation to do so.

\begin{singlespacing}
\subsection*{Reformer John Cotton Dana is credited with many innovations in American museums. What was it specifically that Dana was reacting against.}
\end{singlespacing}

John Cotton Dana was a professional librarian who created the Public Museum of Newark, NJ. With the creation of this museum, Dana also developed a new model of what museums could be. According to Dana, collections had to directly aid members of the local community. He focused on tying the exhibitions at the Public Museum of Newark closely to the needs and interests of the local community. Dana's museum and the exhibits inside it, which were designed as user oriented, were shared freely with the community.

Dana did not see value in private collections, or collections which did not put the local community first. He strongly opposed the Metropolitan Museum of Art located in New York City. This was due to the Met not putting the community first. Additionally, Dana held a strong dislike towards the large staircase leading to the entrance of the Met. According to him, the grandeur of the staircase did not welcome all members of the community.
\end{document}